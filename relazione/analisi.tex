%!TEX TS-program = pdflatex
%!TEX root = progetto_finale.tex
%!TEX encoding = UTF-8 Unicode

\chapter{Analisi}

In questo capitolo, vengono descritti i requisiti funzionali e non funzionali, vale a dire le possibilità offerte al cliente e i requisiti di qualità del sistema.

\section{Requisiti Funzionali}\label{requisiti_funzionali}

I requisiti funzionali comprendono le possibilità offerte all'utente e agli altri programmi.

\begin{enumerate}
	\item Richiesta taxi: tramite questa funzionalità l'utente può chiedere al sistema un veicolo per spostarsi. Essa comprende anche lo spostamento del taxi designato dalla propria posizione a quella del cliente.
	\item Trasporto: il veicolo scelto come più adatto per le esigenze del cliente lo porta dalla posizione di partenza a quella di arrivo.
	\item Utente cambia destinazione: il cliente ha la possibilità di modificare in un qualsiasi momento la propria destinazione, previa notifica del taxi designato. Il sistema si adatta a questa nuova modifica valutandone la possibilità, nel caso in cui il taxi designato non riesca a portarlo nella destinazione scelta questo farà una nuova richiesta di trasporto, in modo completamente trasparente per l'utente. Questa facoltà è permessa solo nel caso in cui l'utente sia l'unico attualmente assegnato al taxi.
	\item Possibili incidenti delle macchine: viene considerata l'eventualità di situazioni in cui un mezzo non è in grado di portare a termine il proprio compito.
	\item Possibilità di cambiamento delle strade: a seguito di incidenti o altri eventi, è possibile che una strada non possa essere percorsa oppure che appaiano nuove strade. Tuttavia si assume che esista sempre un percorso per connettere due diversi punti della città.
\end{enumerate}

Più nello specifico, ogni requisito è caratterizzato da tre componenti: dati in input, dati in output ed effetto desiderato.

\begin{enumerate}
	\item Richiesta taxi
		\begin{itemize}
			\item Input:  $\langle$Posizione iniziale, Posizione finale, id utente$\rangle$
			\item Output: Se presente taxi che possa raggiungere 'posizione finale' -> $\langle$id Taxi designato$\rangle$ \\
						  altrimenti -> avviso assenza taxi disponibile
			\item Effetto: il Taxi con id = 'id Taxi designato' (quello più adatto all'utente, secondo diversi parametri) si sposta dalla sua posizione a 'Posizione iniziale'. Successivamente avviene il Trasporto (vedasi prossimo requisito) di 'id utente' verso la 'Posizione finale'. 'id Taxi designato' potrebbe variare nel caso di incidenti o variazioni della città. Non vi è una garanzia di tempo entro il quale avverrà il Trasporto, imprevisti vari possono ritardare il servizio con tempistiche indefinite, tuttavia nel caso di tali imprevisti il servizio garantisce la presa in carico dello spostamento dell'utente secondo le sue richieste.
		\end{itemize}

	\item Trasporto
		\begin{itemize}
			\item Input:  $\langle$Utente situato in posizione A$\rangle$
			\item Output: $\langle$Utente situato in posizione B$\rangle$
			\item Effetto: la posizione dell'utente dopo un lasso di tempo proporzionale alla lunghezza del tragitto, assumendo assenza di imprevisti che ritardino i tempi, si trova nella posizione B.
		\end{itemize}

	\item Utente cambia destinazione
		\begin{itemize}
			\item Input: $\langle$id utente, id taxi designato, posizione B$\rangle$
			\item Output: Se possibile cambio dest. -> avviso possibilità spostamento \\
						  altrimenti -> avviso impossibilità spostamento con 'id taxi designato'
			\item Effetto: il Taxi al quale è assegnato l'utente valuta se è c'è la possibilità di soddisfarne la richiesta di trasporto al nuovo target = 'posizione B'. Se è nelle facoltà del taxi trasporterà il cliente alla nuova destinazione, altrimenti prenderà in carico la sua richiesta cercando un altro veicolo che possa soddisfare i nuovi bisogni del cliente.
		\end{itemize}
	
	Qui di seguito verranno elencati alcuni requisiti non direttamente interessanti per l'utente ma che sono importanti da definire per la progettazione in quanto rischiano di compromettere il funzionamento del servizio stesso.
	
	% oltre al fatto che rappresentano un interesse indiretto per l'utilizzatore del servizio, 
	\item Rimozione di un taxi
		\begin{itemize}
			\item Input: $\langle$id taxi$\rangle$
			\item Output: $\langle$eliminazione del taxi dalla mappa$\rangle$
			\item Effetto: un generico evento di rottura catastrofico, il taxi scompare dalla città. NB: Si assume che il taxi da rimuovere, ossia con id = 'id taxi', non stia compiendo alcuna operazione per il servizio, ossia che si trovi in una situazione "a riposo", nella quale non trasporta clienti, non si sta dirigendo verso uno di essi e non contribuisce alla ricerca di un taxi designato.
		\end{itemize}
	
	\item Incidente di un taxi
	\begin{itemize}
		\item Input: $\langle$id taxi$\rangle$
		\item Output: $\langle$taxi impossibilitato al movimento$\rangle$
		\item Effetto: un generico evento di rottura non catastrofico, come per esempio una gomma bucata o un problema al motore. La stazione Wi-Fi del taxi continua a funzionare. Nel caso in cui il taxi stia trasportando un cliente, si prenderà la briga di trovare un nuovo taxi per garantire il servizio all'utente.
	\end{itemize}

	\item Modifica della topologia cittadina
		\begin{itemize}
			\item Input: $\langle$topologia cittadina attuale$\rangle$
			\item Output: $\langle$topologia cittadina con strada inagibile oppure con nuova via$\rangle$
			\item Effetto: ogni taxi aggiorna la propria mappa. Se la modifica comprende parti del percorso che il taxi sta percorrendo, il veicolo cerca un percorso alternativo. Se questa nuova strada non è percorribile da questo taxi per via della sua eccessiva lunghezza, esso inoltra una richiesta al sistema di trovare un nuovo taxi in grado di soddisfare i bisogni dell'utente. Nel caso in cui ci siano più utenti serviti dallo stesso taxi, si darà precedenza a quelli che hanno effettuato prima la loro prenotazione. Gli altri verranno notificati dell'impossibilità del trasporto.
		\end{itemize}

\end{enumerate}

\section{Requisiti Non Funzionali} \label{requisiti_non_funzionali}
Sono stati individuati diversi requisiti non funzionali: Adattabilità, Prestazioni e Limiti di tempo. Come conseguenza del CAP Theorem sono stati trovati, inoltre, altri requisiti.
Le trasparenze sono già state discusse nell'introduzione al punto \ref{intro_trasparenze}.

\subsection{Vari Requisiti}
\begin{itemize}
	\item Adattabilità: il sistema prevede di poter gestire l'aumento o diminuzione dei veicoli disponibili e delle richieste dei clienti.
	\item Prestazioni: si vuole utilizzare il numero di messaggi minimo per le comunicazioni tra utenti e macchine e tra veicoli per ottenere una risposta celere. I tempi di applicazione degli algoritmi devono essere bassi per garantire tempi di risposta rapidi considerato che la città può avere molti nodi, visto il requisito di scalabilità. Si vuole inoltre garantire che la macchina scelta sia la più veloce tra quelle in gioco nell'utilizzo dell'algoritmo.
	\item Limiti di tempo: data la possibilità della disconnessione del grafo delle comunicazioni tra veicoli, è necessario definire un limite temporale per l'elezione del leader.
\end{itemize}

\subsection{CAP Theorem}
Come conseguenza del CAP Theorem, e quindi dell'impossibilità di garantire consistenza, disponibilità e tolleranza alle partizioni assieme, si è valutato, per il progetto in esame, garantire disponibilità e tolleranza alle partizioni.

Riguardo la consistenza, è necessario considerare due grafi: mappa della città e mappa delle comunicazioni tra i veicoli. 

\subsubsection{Consistenza}
La consistenza riguardo le mappe non può essere garantita poiché i nodi in movimento hanno una visione locale dei nodi vicini e delle eventuali modifiche alle strade. Per semplicità del progetto, tuttavia, si assume che la mappa della città sia sempre aggiornata per ogni macchina. In un caso reale, sarebbero i veicoli a comunicare agli altri le strade non più agibili o quelle nuove disponibili. I clienti, infine, hanno una visione parziale delle macchine disponibili poiché possono comunicare solo con quella più vicina a loro.

\subsubsection{Disponibilità}
Essa è garantita dal fatto che si assume l'utente possa sempre comunicare con almeno un veicolo, il quale è l'iniziatore dell'algoritmo di elezione, pertanto nel caso peggiore non viene trovato un vincitore e l'utente viene sempre notificato sulla disponibilità del servizio. Di questo si era già accennato nell'introduzione nelle problematiche distribuite \ref{problematiche_distribuite}.

\subsubsection{Tolleranza alle Partizioni}
Il partizionamento della mappa delle macchine, la rete del sistema distribuito, non rappresenta un problema per il corretto funzionamento del servizio:  soltanto nel caso di totale fallimento dei veicoli il sistema non risponde correttamente, visto e considerato che per la notifica al cliente è sufficiente una partizione di auto nelle sue vicinanze. Di questo si era già accennato nell'introduzione nelle parte relativa agli algoritmi \ref{intro_algo}.
