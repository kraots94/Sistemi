%!TEX TS-program = pdflatex
%!TEX root = progetto_finale.tex
%!TEX encoding = UTF-8 Unicode

\chapter{Progetto} \label{progetto}

This chapter is devoted to the description of the general architectures, and specific algorithms.

\section{Architettura Logica}
Describe the components of your systems: modules/objects/components/services. For each component, describe the functionalities it implements, and by who is used.

\section{Protocolli e algoritmi}
Communication between components. UML sequence diagrams go here. Also, put here a detailed description of distributed algorithms used to solve specific problems of the project.

\section{Architettura fisica e Distribuzione}
Which nodes and platforms involved, and where each component is deployed.

\section{Piano di Sviluppo}
Since it is diffcult to predict just how hard implementing a new system will be, you should formulate as a set of "tiers", where the basic tier is something youre sure you can complete, and the additional tiers add more features, at both the application and the system level.

Lo sviluppo dell'applicazione verrà suddiviso in diverse versioni via via estese con le diverse funzionalità. In particolare si seguirà questo ordine:
\begin{enumerate}
	\item Ogni macchina potrà parlare con tutte le altre macchine; l'utente invia la richiesta di trasporto alla macchina più vicina a lui; se un veicolo è impegnato con un cliente non partecipa alla selezione del leader per il trasporto; la comunicazione tra i veicoli è diretta, quindi ogni taxi può parlare con chiunque.
	\item Vengono aggiunte le colonnine di ricarica, alle quali le macchine devono far rifornimento se hanno esaurito le batterie; le macchine possono guastarsi; l'utente può decidere di cambiare destinazione.
	\item La macchina partecipa all'elezione anche se al momento è già impegnata, tuttavia si considera disponibile solo dopo aver compiuto il tragitto già attivo, rottura delle strade ma senza disconnessione del grafo stradale.
	\item Implementazione del car-sharing fino a 3 clienti.
	\item Le macchine possono comunicare solo con quelle vicine, viene fornita al cliente una lista di possibili candidati.
	\item Le connessioni tra le macchine possono perdere andare perse mentre si elegge il leader.
	\item panino alla guida
\end{enumerate}