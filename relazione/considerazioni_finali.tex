%!TEX TS-program = pdflatex
%!TEX root = progetto_finale.tex
%!TEX encoding = UTF-8 Unicode

\chapter{Considerazioni Finali}

Non essendo stato richiesto nella consegna, non sono state gestite le situazioni in cui un processo termini bruscamente senza notificare le entità con cui sta comunicando. Si pensi alla gerarchia tra ascoltatore e automi sottostanti dell'entità macchina. Per risolvere questo problema si sarebbe potuto utilizzare i ``supervisor'' di erlang impostando l'ascoltatore come supervisore: in tal modo egli avrebbe garantito un respawn dei processi caduti.

Per una migliore comprensione dello stato durante le fasi di debugging delle diverse entità all'interno dell'ambiente, sarebbe stato opportuno sviluppare anche un'interfaccia grafica che mostri in tempo reale la posizone e le diverse interazioni tra le entità. Non è stata fatta poiché è stata presta maggiore attenzione allo sviluppo delle funzionalità che il sistema doveva offrire. 

Similmente, per un'applicazione reale del sistema sarebbe stato opportuno sviluppare un interfaccia grafica per l'applicazione dell'utente. Come precedentemente detto, l'attenzione è stata concentrata sulla gestione degli eventi e delle diverse casistiche che il sistema deve supportare rispetto alle interfaccia di utilizzo.
