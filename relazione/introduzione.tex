%!TEX TS-program = pdflatex
%!TEX root = progetto_finale.tex
%!TEX encoding = UTF-8 Unicode

\chapter{Introduzione}

Il problema in esame consiste nella gestione dei taxi elettrici e assegnazione di essi agli utenti che ne fanno richiesta per spostarsi all'interno di una città. Si vuole creare un sistema distribuito in modo che non sia necessario un server centralizzato. Con questa modalità, sono le vetture che si accordano tra di loro, attraverso un meccanismo di comunicazione wireless, per decidere quale sia il miglior veicolo che possa soddisfare la richiesta del cliente. Allo stesso modo, egli richiede il servizio tramite un'applicazione dedicata che comunica direttamente con una di queste macchine, con lo un metodo di comunicazione senza fili dedicato. 

La soluzione proposta consiste in:
\begin{enumerate}
	\item L'utente, tramite l'app, invia la richiesta di spostamento al taxi a lui più vicino indicando la propria posizione e destinazione.
	\item La vettura riceve la richiesta e, comunicando con gli altri taxi, viene deciso quale veicolo sia il più adatto a soddisfare il bisogno dell'utente.
	\item Per decidere quale sia la vettura più adatta, si prendono in considerazione diversi parametri, tra cui: distanza dall'utente, carica rimanente e se la vettura al momento è già occupata.
	\item Notifica al cliente da parte della macchina di essere stata scelta per il trasporto.
	\item Il veicolo scelto si sposta verso l'utente e lo trasporta verso la direzione richiesta.
\end{enumerate}

\newpage

\section{Problemi del sistema distribuito} \label{problematiche_distribuite}

Alcuni problemi che potrebbero emergere in un contesto distribuito, interessanti da affrontare nella costruzione del sistema, sono: la scelta del miglior taxi da assegnare al cliente, la possibilità di rottura di un veicolo, il cambio destinazione di un utente, il cambiamento della topologia della città.

Si potrebbe pensare che nella realtà, a seguito dell'inagibilità di una strada, una parte della città sia isolata dal resto. Questa situazione non viene presa in considerazione poiché, in uno scenario reale, la disconnessione della città è praticamente impossibile visto che sono presenti strade alternative che garantiscono un percorso.

Un'altra possibile situazione è quella in cui un veicolo, a seguito di uno spostamento, si isoli e non sia più in grado di comunicare con le altre vetture. Questa possibilità è realistica, tuttavia non è problematica visto che gli algoritmi utilizzati non prevedono una connessione completa del grafo di comunicazione delle macchine.

Per utilizzare il servizio, l'applicazione utilizzata dagli utenti comunica con le vetture tramite delle speciali celle riservate alla comunicazione cliente-taxi, simili a dei ripetitori. Queste strutture hanno come unico scopo quello di permettere lo scambio di messaggi tra applicazione e taxi in modo bidirezionale. Questo permette di evitare la situazione in cui un cliente non possa utilizzare il servizio.

\section{Componenti del sistema}

Le componenti individuate nella modellazione del problema sono principalmente tre:

\begin{itemize}
	\item Macchina: rappresenta una vettura che ha i compiti sopra descritti.
	\item Utente: rappresenta una persona che vuole spostarsi all'interno della città. Comunica con un veicolo e aspetta che arrivi la macchina designata.
	\item Ambiente: interviene su macchine, strade e utenti emulando le comunicazioni wireless, gli eventi di rottura delle macchine e inagibilità delle strade, inserisce utenti e macchine nella città.
\end{itemize}

Per chiarezza, l'ambiente sopperisce alla mancanza di fattori presenti in un contesto reale: schede di rete che permettano connessioni wireless tra macchine, eventi casuali che possano causare diversi tipi di errori nel sistema, sostituzione di una macchina, variazione dei clienti. In tal senso, l'ambiente non offre nessun servizio alle altre entità equiparabile a quello di un server, se non quelli ottenibili tramite i fattori appena descritti.

\section{Architettura}

L'architettura scelta per il progetto è quella peer-to-peer, quindi le macchine comunicano fra di loro in una rete mesh, con nodi idem-potenti, attraverso scambi di messaggi tramite connessioni affidabili (ossia reti che utilizzano ack per una conferma di ricezione e ritrasmissione in caso di errore).

\section{Trasparenze}\label{intro_trasparenze}
Una trasparenza è un aspetto del sistema distribuito che viene nascosto all'utente. Esse vengono discusse nella parte relativa ai requisiti non fuzionali.

\section{Algoritmi} \label{intro_algo}

Per la selezione del veicolo migliore da inviare al cliente che richiede il servizio, è stato scelto un algoritmo di elezione che seleziona il leader secondo la minimizzazione di alcuni parametri, come già discusso nella soluzione proposta. Un sottoinsieme di veicoli, i più vicini all'utente, concordano su quale tra di essi sia da inviare con un tipo di algoritmo Wave, vale a dire l'Echo Algorithm.

\section{Testing del sistema}

Per testare il sistema, l'entità Ambiente utilizzerà un grafo casuale, precedentemente generato, rappresentante una città, e inserirà in esso un gran numero di macchine e utenti, i quali sottoporranno richieste di spostamenti, inoltre l'entità introdurrà una componente di stress attraverso eventi casuali come: rottura di strade, rimozione di veicoli, variazioni richieste utenti, rottura di veicoli che però continuano ad essere nodi attivi nella rete p2p. 

\newpage

\section{Fasi dello sviluppo}
Lo sviluppo del progetto seguirà le seguenti fasi:
\begin{enumerate}
	\item Selezione delle diverse entità che interagiscono all'interno del sistema.
	\item Analisi del problema e creazione di diversi casi d'uso per comprendere le possibili problematiche da modellare.
	\item Specificazione delle comunicazioni che possono avvenire tra le diverse entità.
	\item Scelta degli algoritmi per la risoluzione dei problemi: 
		\begin{itemize}
			\item comunicazione cliente-macchina.
			\item comunicazione macchina-macchina per elezione.
			\item comunicazione macchina-cliente per assegnazione.
			\item comunicazione ambiente-entità presenti.
			\item calcolo del percorso di costo minimo.
		\end{itemize}
	\item Strutturazione dell'implementazione in Erlang.
	\item Implementazione
	\item Test del sistema e validazione.
\end{enumerate}

