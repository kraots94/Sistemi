%!TEX TS-program = pdflatex
%!TEX root = progetto_finale.tex
%!TEX encoding = UTF-8 Unicode

\chapter{Validazione}

In questo capitolo vengono fornite le istruzioni per l'esecuzione del codice, i test effettuati con relativi screenshots ed infine confronto di ciò che il sistema fornisce in più o in meno rispetto ai requisti definiti al capitolo 2 \ref{requisiti_funzionali}.

\section{Compilazione ed esecuzione del codice}
Per lo sviluppo del generatore della mappa della città è stata utilizzata la versione 3.8.3 disponibile al sito \url{https://www.python.org/downloads/}.
Per lo sviluppo e testing del progetto è stata utilizzata la versine 23.1 di erlang, disponibile al sito \url{https://www.erlang.org/downloads}.

Si può utilizzare le mappe già presenti nella cartella ``map'' oppure generarne di nuove eseguendo lo script python con un apposito interprete.
\subsection{Istruzioni per la compilazione}\label{istruzioni_compilazione}
Per poter eseguire il codice è necessario svolgere i seguenti passi:
\begin{enumerate}
	\item Scompattare il contenuto della cartella in una posizione a scelta.
	\item Aprire un terminale e spostari all'interno della cartella root del progetto.
	\item Eseguire il comando \lstinline |make:all().| per compilare i file. Esso crea all'interno della cartella ``ebin'' i file necessari all'esecuzione del codice.
	\item Eseguire il comando \lstinline |cd("ebin").| per spostarsi all'interno della cartella con i beam appena generati.
	\item Eseguire il comando \lstinline |PID_ENV = main:start_project().| per avviare il progetto.
\end{enumerate}
A questo punto l'ambiente è stato avviato e ci si troverà davanti a una situazione simile a questa:

\begin{lstlisting}
up_to_date
3> cd("ebin").
f:/progetti/Sistemi/ebin
ok
4> PID_ENV = main:start_project().
Starting Project
Devs: Alessandro, Angelo
Mails: forgiarini.alessandro@spes.uniud.it, andreussi.angelo@spes.uniud.it
<0.220.0>
5> ["Env"] {<0.220.0>} - "Environment Created"
\end{lstlisting}

\subsection{Esecuzione del codice}\label{esecuzione_codice}
Il modo di interazione attuale tra utente reale e sistema prevede di eseguire i diversi comandi forniti dall'entità ``Environment''. Essa fornisce un insieme di API che permettono di simulare le diverse situazioni che possono venirsi a creare in un contesto reale. Questo si è reso necessario poiché è stato creato un ambiente simulativo della realtà.

Per via della convenzione utilizzata, tutti i nomi sono stringe, siano essi di nodi o di entità. Per passarli come parametro è necessario racchiuderli tra virgolette, per esempio: "aa".

Come già spiegato nel punto \ref{componenti_grafo_citta}, i nomi dei nodi sono delle stringhe del tipo ``aa, ab, ...''.

I nomi delle macchine sono del tipo ``c1, c2, ...''

I nomi degli utenti sono del tipo ``u1, u2, ...''

Per invocare i comandi,  è necessario utilizzare la seguente sintassi dalla console di Erlang:
\begin{lstlisting}
environment:nome_funzione(PID_ENV, Param1, ...).
\end{lstlisting}
Di seguito un elenco con spiegazione dei metodi forniti:
\begin{itemize}
	\item \lstinline |triggerEvent(PID_ENV, ID)|

	Permette di scatenare nell'ambiente l'evento con ID pari a quello passato come parametro. Un elenco degli eventi disponibili è al punto \ref{env_events}.
	\item \lstinline |triggerChangeDestUser(PID_ENV, UserName)|
	
	Permette di cambiare la destinazione attuale dell'utente passato come parametro con una destinazione casuale.
	\item \lstinline |triggerChangeDestUser(PID_ENV, UserName, NewTarget)|
	Permette di cambiare la destinazione attuale dell'utente passato come parametro con una destinazione specifica.
	\item \lstinline |triggerCarCrash(PID_ENV, CarName)|
	
	Scatena l'evento di ``crash'' di una macchina con conseguente notifica agli utenti in coda e impossibilità di muoversi fino all'evento di riparazione.
	\item \lstinline |triggerFixCar(PID_ENV, CarName)|

	Scatena l'evento di ``fix'' di una macchina incidentata con conseguente possibilità di vittoria delle elezioni e trasporto degli utenti.
	\item \lstinline |triggerCarRemove(PID_ENV, CarName)|

	Se la macchina passata come parametro è nello stato ``idle'' la rimuove dal sistema, altrimenti notifica che non è stato eseguito nulla.
	\item \lstinline |enableAutoEvents(PID_ENV)|

	Abilita gli eventi casuali automatici.
	\item \lstinline |disableAutoEvents(PID_ENV)|

	Disabilita gli eventi casuali automatici.
	\item \lstinline |printGpsState(PID_ENV)|
	
	Stampa lo stato interno del server gps. In tal modo è possibile sapere in quel momento dove si trovano tutte le entità.
	\item \lstinline |printSelfState(PID_ENV)|

	Stampa lo stato interno dell'entità environment.
	\item \lstinline |printCars(PID_ENV)|

	Stampa il dizionario delle macchine, vale a dire tutte le associazioni Nome - Pid nel caso delle macchine.
	\item \lstinline |printUsers(PID_ENV)|

	Stampa il dizionario degli utenti, vale a dire tutte le associazioni Nome - Pid nel caso degli utenti.
	\item \lstinline |spawnCars(PID_ENV, N)|

	Crea un numero N di entità macchina in posizione casuale all'interno della mappa.
	\item \lstinline |spawnUsers(PID_ENV, N)|

	Crea un numero N di entità utente in posizione casuale all'interno della mappa con una destinazione casuale.
	\item \lstinline |spawnCar(PID_ENV, StartingPos)|

	Crea una macchina in una determinata posizione della mappa.
	\item \lstinline |spawnUser(PID_ENV, From, To)|	

	Crea un utente con una determinata richiesta di trasporto. Egli verrà creato nel nodo ``From''.
\end{itemize}

\section{Test eseguiti}\label{test_eseguiti}
Per poter valutare se il sistema soddisfa i requisti proposti nella parte \ref{requisiti_funzionali}, si è deciso di effettuare diversi test. La mappa utilizzata è quella presente nella parte di presentazione \ref{componenti_grafo_citta}.

\subsection{Test - Un cliente una macchina}
Il primo di essi consiste nella semplice generazione di un cliente e di una macchina casuale. Di seguito i risultati ottenuti:

\begin{lstlisting}
4> environment:spawnCars(PID_ENV, 1).
{spawn,cars,1}
5> ["Car"] {"c1"} - Car ready in position ["b"]
5> environment:spawnUsers(PID_ENV, 1).
{spawn,users,1}
6> ["User"] {"u1"} - I'm a new user, i'm going from ["c"] to ["h"]
6> ["Car"] {"c1"} - Car Stops for client: ["c","h"] | Total Time for client: 34
6> ["User"] {"u1"} - Taxi ["c1"] is serving me with 14 time to wait
6> ["User"] {"u1"} - Taxi ["c1"] is arrived in my position!
6> ["Car"] {"c1"} - Arrived in user target position ["h"]
6> ["User"] {"u1"} - I'm arrived in my target position ["h"]
6> ["User"] {"u1"} - "I made my trip, goodbye."
\end{lstlisting}

\subsection{Test - Più clienti una macchina}
Il secondo caso analizzato è il caso di più utenti che richiedono ad un unico taxi di essere serviti. Dopo aver creato un taxi vengono istanziati tre clienti. Nell'appendice al punto \ref{tre_utenti_una_macchina_test} è presente il log di come il sistema ha gestito il test. 

Ci sono diverse cose da notare in esso ripercorrendo il log:
\begin{itemize}
	\item Il messaggio di begin election dell'utente ``u1'' è il primo giunto al taxi poiché gli altri due hanno stampato il messaggio di errore relativo al caso in cui la macchina a cui viene chiesto il servizio è già nello stato di running election.
	\item Dopo la comunicazione da parte della macchina all'utente ``u1'' dei propri tempi di percorrenza, l'utente ``u3'' ha esaurito la propria attesa casuale per il nuovo tentativo di richiesta del servizio e, dopo aver inviato la richiesta, riceve la risposta con i tempi. 
	\item L'utente ``u2'', essendo l'ultimo ad effettuare la richiesta, non viene soddisfatto e resta nello stato di waiting election. La macchina non risponderà in modo positivo finché non possiederà abbastanza batteria.
	\item Dopo aver trasportato sia ``u1'' che ``u3'', la macchina continua a fare i calcoli per capire se può soddisfare ``u2'' tuttavia la bassa batteria le impedisce di vincere. Essendo nello stato di ``idle'', tuttavia, dopo un numero di tick pari a 20, viene abilitata la carica solare che ricarica ad ogni tick ricevuto un'unità di batteria.
	\item si può notare infatti come il messaggio stampatao da parte della macchina ``Triggered Solar Charghing'' notifichi l'inizio della ricarica.
	\item Non appena sufficiente batteria è stata ricaricata, la macchina vince l'elezione e trasporta l'utente ``u2'' a destinazione.
	\item Si può notare come, dopo il trasporto di ``u2'' e il viaggio alla colonnina, la macchina abbia zero batteria e inizia la ricarica normale.
\end{itemize}

\subsection{Test - Un utente e diverse macchine}
Il terzo test mostra il caso in cui sono presenti nell'ambiente più macchine e viene ricevuta la richiesta di un utente. All'appendice al punto \ref{log_one_user_more_cars} si trova il log.

\subsection{Test - Utente cambia destinazione}
Il quarto test consiste nell'evento di cambio destinazione da parte dell'utente. Il log è presente nell'appendice al punto \ref{log_change_dest}.

Si può notare come il percorso della macchina cambi da prima a dopo la richiesta dell'utente.

\subsection{Test - Evento Crash e Fix}
Il quinto test mostra come il sistema reagisce in base al crash di una macchina. Il log è presente nell'appendice al punto \ref{car_crash_fix_event}.

In seguito alla chiamata di sistema che causa l'evento ``crash'' di una macchina, essa notifica a tutti gli utentei in coda il fatto di non poter più proseguire, infatti si vedono le notifiche da parte degli automi di questo. 

Dopo un tempo random si è deciso di riparare la macchina. Le applicazioni degli utenti sono riuscite quindi a riottenere il servizio da parte della macchina e garantire il servizio.