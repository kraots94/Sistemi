%!TEX TS-program = pdflatex
%!TEX root = progetto_finale.tex
%!TEX encoding = UTF-8 Unicode

\chapter{Validazione}

In questo capitolo vengono fornite le istruzioni per l'esecuzione del codice, i test effettuati con relativi screenshots ed infine confronto di ciò che il sistema fornisce in più o in meno rispetto ai requisti definiti al capitolo 2 \ref{requisiti_funzionali}.

\section{Compilazione ed esecuzione del codice}
Per lo sviluppo del generatore della mappa della città è stata utilizzata la versione 3.8.3 disponibile al sito \url{https://www.python.org/downloads/}.
Per lo sviluppo e testing del progetto è stata utilizzata la versine 23.1 di erlang, disponibile al sito \url{https://www.erlang.org/downloads}.

Si può utilizzare le mappe già presenti nella cartella ``map'' oppure generarne di nuove eseguendo lo script python con un apposito interprete.
\subsection{Istruzioni per la compilazione}\label{istruzioni_compilazione}
Per poter eseguire il codice è necessario svolgere i seguenti passi:
\begin{enumerate}
	\item Scompattare il contenuto della cartella in una posizione a scelta.
	\item Aprire un terminale e spostari all'interno della cartella root del progetto.
	\item Eseguire il comando \lstinline |make:all().| .
	\item Eseguire il comando \lstinline |cd("ebin").| per spostarsi all'interno della cartella con i beam appena generati.
	\item Eseguire il comando \lstinline |PID_ENV = main:start_project().| per avviare il progetto.
\end{enumerate}
A questo punto l'ambiente è stato avviato e ci si troverà davanti a una situazione simile a questa:

\begin{lstlisting}
up_to_date
3> cd("ebin").
f:/progetti/Sistemi/ebin
ok
4> PID_ENV = main:start_project().
Starting Project
Devs: Alessandro, Angelo
Mails: forgiarini.alessandro@spes.uniud.it, andreussi.angelo@spes.uniud.it
<0.220.0>
5> ["Env"] {<0.220.0>} - "Environment Created"
\end{lstlisting}

\subsection{Esecuzione del codice}


\section{Test eseguiti}

\section{Controllo requisiti}