%!TEX TS-program = pdflatex
%!TEX root = progetto_finale.tex
%!TEX encoding = UTF-8 Unicode

\chapter{Implementazione}

Il linguaggio scelto per l'implementazione del progetto è Erlang. Esso è stato selezionato per i seguenti motivi: fornisce nativamente il supporto alla gestione di messaggi e scambio di essi tra diverse entità; permette di creare in poche righe di codice diversi beam a cui poter richiedere dei servizi; grazie alla libreria gen\_statem è possibile creare facilmente gli automi.

In questo progetto non utilizzate piattaforme esterne di alcun genere poiché si assume che l'autenticazione tra utente e applicazione per il servizio sia già stata effettuata.

Essendo un'architettura peer to peer, non è necessaria alcuna piattaforma a parte l'autenticazione cliente - servizio.

\section{Generazione dell'ambiente virtuale}
Il problema della modellazione di una città è stato risolto con un grafo pesato connesso non orientato: si assume che se due nodi sono connessi è perché vi è una strada tra di essi ed è bidirezionale. Il peso indica quanto costa attraversarla. 

\subsection{Componenti del grafo}
Ogni nodo del grafo possiede quattro proprietà:
\begin{itemize}
	\item attributo nome: generato del tipo ``aa'', ``ab'', ``ac'', ... incrementale per ogni nodo.
	\item attributo id: numero incrementale per ogni nodo a partire da 0. Utilizzato per la definizione del grafo della città.
	\item Coordinate X e Y: definiscono la posizione del nodo all'interno del piano cartesiano rappresentante la città.
\end{itemize}

Per quanto riguarda gli archi, essi posseggono tre proprietà:
\begin{itemize}
	\item Nodo 1 - Nodo 2 : indica quali nodi collega.
	\item Peso: indica il peso dell'arco, calcolato in funzione della distanza tra i nodi che collega l'arco.
\end{itemize}

Infine vengono selezionati alcuni nodi tra quelli del grafo che conterranno le colonnine di ricarica. Per far questo è stato applicato il seguente algoritmo:
\begin{enumerate}
	\item In base al numero totale delle colonnine, suddivisione del grafo in parti uguale.
	\item Selezione di un nodo casuale all'interno della parte creata.
	\item Per fornire una distribuzione abbastanza omogenea, ogni zona del grafo conterrà un'unica colonnina. Ad un nodo casuale della zona viene assegnata la colonnina.
\end{enumerate}

\subsection{File generati}
Il linguaggio utilizzato per la generazione della città è python. All'interno del file ``map\_generator'' è possibile impostare i diversi parametri per la generazione della città, vale a dire:
\begin{itemize}
\item WEIGHT\_MIN: Peso minimo degli archi.
\item WEIGHT\_MAX:  Peso massimo degli archi.
\item TOTAL\_NODES: Totale numero di nodi del grafo.
\item TOTAL\_EDGES:  Totale numero di archi del grafo.
\item TOTAL\_CHARGING\_COLS:  Totale numero delle colonnine di ricarica nel grafo.
\item CARTESIAN\_SIDE:  Lato del quadrato cartesiano utilizzato per la mappa.
\end{itemize}

Per quanto riguarda il numero di colonnine, esso non è garantito essere rispettato per il seguente motivo: in base al numero impostato, il piano cartesiano viene suddiviso in parti uguali e, se i nodi non sono molto densi, può capitare che alcune di esse siano vuote. Per fornire una distribuzione abbastanza omogenea delle colonnine, esse vengono posizionate una per parte, vale a dire che in ogni zona, anche se sono presenti più nodi, solo uno di essi conterrà la colonnina. Per questi motivi nel caso in cui una parte delle colonnine occupi già tutti le zone disponibili l'altra parte non verrà posizionata.

Lo script in python crea cinque file:
\begin{itemize}
	\item ``map.pdf'': 
	\item ``map.png'':
	\item ``city\_map\_nodes.dat'':
	\item ``city\_map\_graph.dat'':
	\item ``city\_map\_charging\_cols.dat'':
\end{itemize}

\section{Entità presenti}

\section{Fornitori di servizi}

\section{Automa Elezione}

\section{Interazione tra i diversi moduli}


Istruzioni per la compilazione ed esecuzione
Scompattare il contenuto della cartella in una posizione a scelta. 
Avviare erlang e posizionarsi all'interno della root del progetto. 
Se assente, creare la cartella ``ebin'' sempre nella root.
Eseguire il comando
\lstinline|make:all().|
Spostarsi nella cartella "ebin" tramite il comando
\lstinline|cd("ebin").|
\lstinline| PID_ENV = main:start_project().|
Viene avviato l'environment 



Details about the implementation: every choice about platforms, languages, software/hardware, middlewares, which has not been decided in the requirements.
Important choices about implementation should be described here; e.g., peculiar data structures.