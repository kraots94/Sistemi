%!TEX TS-program = pdflatex
%!TEX root = progetto_finale.tex
%!TEX encoding = UTF-8 Unicode

\chapter{Implementazione}

Il linguaggio scelto per l'implementazione del progetto è Erlang. Esso è stato selezionato per i seguenti motivi: fornisce nativamente il supporto alla gestione di messaggi e scambio di essi tra diverse entità; permette di creare in poche righe di codice diversi beam a cui poter richiedere dei servizi; grazie alla libreria gen\_statem è possibile creare facilmente gli automi.

In questo progetto non utilizzate piattaforme esterne di alcun genere, si assume che l'autenticazione tra utente e applicazione per il servizio sia già stata effettuata.

Essendo un'architettura peer to peer, non è necessaria alcuna piattaforma a parte l'autenticazione cliente - servizio.

\subsection{Generazione dell'ambiente virtuale}

\subsection{Entità presenti}

\subsection{Fornitori di servizi}

\subsection{Automa Elezione}

\subsection{Interazione tra i diversi moduli}


Details about the implementation: every choice about platforms, languages, software/hardware, middlewares, which has not been decided in the requirements.
Important choices about implementation should be described here; e.g., peculiar data structures.